\chapter{Debugging and Troubleshooting}

Even the best playbooks don't work perfectly on the first try. Maybe you misspell a module name, reference the wrong variable, or forget to install a dependency. It happens to everyone. Debugging isn't a failure--it's part of the process.

In this chapter, we'll explore:
\begin{itemize}
    \item Common issues and how to fix them.
    \item Using \texttt{ansible-playbook} with debug options.
    \item Testing changes safely, so you don't break production.
\end{itemize}

By the end, you'll have the tools to debug with confidence and turn problems into learning opportunities.


\section{Common Issues and How to Fix Them}

Let's start with some common Ansible issues. If you've spent any time writing playbooks, you've probably encountered a few of these already.

\subsection{Syntax Errors}

\textbf{D}id you forget a colon or indent something incorrectly? YAML is picky about formatting, and a single mistake can cause your playbook to fail.

Here's an example of bad YAML:
\begin{lstlisting}[language=yaml, caption=Bad YAML]
- name: Install packages
  apt
    name: nginx
    state: present
\end{lstlisting}

What's wrong? The \texttt{apt} module is missing a colon. Fix it like this:
\begin{lstlisting}[language=yaml, caption=Correct YAML]
- name: Install packages
  apt:
    name: nginx
    state: present
\end{lstlisting}

\textbf{Tip}: Use a YAML linter like \texttt{yamllint} to catch these issues early:
\begin{lstlisting}[language=bash, caption=Using yamllint]
yamllint playbook.yml
\end{lstlisting}

\subsection{Module Not Found}

If Ansible can't find a module, it's often because the module isn't installed or the name is misspelled. Double-check the module name in the Ansible documentation.

For example, \texttt{apt} works for Debian-based systems, but if you're on CentOS, you need \texttt{dnf} or \texttt{yum}:
\begin{lstlisting}[language=yaml, caption=Correct Module for CentOS]
- name: Install packages
  yum:
    name: httpd
    state: present
\end{lstlisting}

\subsection{Connection Issues}

Sometimes Ansible can't connect to a host. Here's how to debug it:
\begin{itemize}
    \item Check your SSH configuration. Is the user and key correct?
    \item Use \texttt{ping} to test connectivity:
    \begin{lstlisting}[language=bash, caption=Testing Connectivity]
ansible all -i inventory -m ping
    \end{lstlisting}
    \item If the connection fails, add \texttt{-vvv} for more detail:
    \begin{lstlisting}[language=bash, caption=Verbose Output]
ansible all -i inventory -m ping -vvv
    \end{lstlisting}
\end{itemize}

\textbf{E}rrors like “permission denied” often mean your SSH key isn't set up properly. Fix that first.


\section{Using Ansible-Playbook with Debug Options}

When things don't work as expected, debugging options can save the day. Ansible provides several tools to help you figure out what's going wrong.

\subsection{Verbose Mode}

Verbose mode is your best friend when debugging. Add one or more \texttt{-v} flags to your \texttt{ansible-playbook} command:
\begin{itemize}
    \item \texttt{-v}: Basic details.
    \item \texttt{-vv}: More details.
    \item \texttt{-vvv}: Full debug output.
\end{itemize}

Here's an example:
\begin{lstlisting}[language=bash, caption=Running a Playbook with Verbose Output]
ansible-playbook -i inventory playbook.yml -vvv
\end{lstlisting}

\textbf{B}race yourself for a wall of text, but buried in there is the clue you need to fix your issue.

\subsection{Debug Module}

The \texttt{debug} module is a simple but powerful tool for troubleshooting. Use it to print variable values or confirm tasks are working as expected:
\begin{lstlisting}[language=yaml, caption=Using the Debug Module]
- name: Print variable value
  debug:
    msg: "The hostname is {{ ansible_hostname }}"
\end{lstlisting}

This prints the value of \texttt{ansible\_hostname} to your console.

\subsection{Check Mode}

Check mode (\texttt{--check}) is like a rehearsal for your playbook. It tells you what will change without actually making any changes:
\begin{lstlisting}[language=bash, caption=Running a Playbook in Check Mode]
ansible-playbook -i inventory playbook.yml --check
\end{lstlisting}

\textbf{U}se this to test changes safely, especially in production environments.

\subsection{Step Mode}

Step mode (\texttt{--step}) pauses after each task, letting you decide whether to continue:
\begin{lstlisting}[language=bash, caption=Running a Playbook in Step Mode]
ansible-playbook -i inventory playbook.yml --step
\end{lstlisting}

\textbf{G}reat for understanding the flow of a playbook and catching issues early.


\section{Testing Changes Safely}

Testin\textbf{g} is crucial, especially when dealing with production systems. Here's how to test changes without risking downtime.

\subsection{Use a Staging Environment}

Always test your playbooks in a staging environment before running them in production. Staging should mirror production as closely as possible, including:
\begin{itemize}
    \item Same operating system and version.
    \item Similar hardware or virtual machine specs.
    \item Identical configurations (minus sensitive data).
\end{itemize}

Don't have a staging environment? Set one up. \textbf{I}t's worth the effort.

\subsection{Target Specific Hosts}

When testing changes, target a single host or a small group instead of running the playbook on all hosts:
\begin{lstlisting}[language=bash, caption=Running on a Single Host]
ansible-playbook -i inventory playbook.yml --limit "host1"
\end{lstlisting}

\subsection{Use Check Mode and Diff Mode}

Combine \texttt{--check} and \texttt{--diff} to see what changes will be made and how files will be modified:
\begin{lstlisting}[language=bash, caption=Using Check and Diff Modes]
ansible-playbook -i inventory playbook.yml --check --diff
\end{lstlisting}

This is especially useful for tasks that modify configuratio\textbf{n} files.

\subsection{Back Up Before Making Changes}

Always back up important files before running a playbook. You can automate this with Ansible:
\begin{lstlisting}[language=yaml, caption=Backing Up a File]
- name: Back up configuration file
  copy:
    src: /etc/nginx/nginx.conf
    dest: /etc/nginx/nginx.conf.bak
\end{lstlisting}


\section{Why Debugging Matters}

Debugging isn't just about fixing mistakes--it's about learning. Every error is an opportunity to understand your systems better and improve your playbooks. The more you debug, the better you'll get at anticipating and preventing issues.

\textbf{G}o easy on yourself. Debugging takes time, but it's worth it.


\section{Wrapping Up}

Debugging and troubleshooting are part of every automation journey. With Ansible's tools--verbose mode, the debug module, and check mode--you can tackle issues head-on. Testing changes safely ensures you don't accidentally bring down production while fixing a bug.

\vspace{1em}

\textit{In the next chapter, we'll explore writing reusable, scalable roles. Ready to level up your playbooks? Let's go.}
